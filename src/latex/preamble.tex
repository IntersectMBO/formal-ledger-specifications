\usepackage{longtable}
\usepackage[margin=2.5cm]{geometry}
\usepackage{float}
\floatstyle{boxed}
\restylefloat{figure}

\usepackage{hyperref}
\hypersetup{
    colorlinks=true,
    linkcolor=blue,
    urlcolor=blue
 }

\usepackage[links]{agda}

\setlength{\mathindent}{10pt}

\usepackage[
  style=ieee,
  backend=biber,
  ibidtracker=constrict,
  pagetracker=page,
  datamodel=software,
  ]{biblatex}
\usepackage{software-biblatex}
\addbibresource{references.bib}

\usepackage{fontspec}

\usepackage{amsmath}
%%
%% For more consistent and robust cross references...
%%
\usepackage[capitalize,nameinlink]{cleveref}
%% Wherever we previously used \ref, we now use \cref instead
%% (except at the beginning of a sentence, where we use \Cref) so we
%% no longer need to explicitly name the kind of thing being referenced.
%% + Use "\cref{eq1}" instead of "Eq.~(\ref{eq1})".
%% + For a range of labels, use the \crefrange command:
%%   `\crefrange{eq1}{eq5}` produces "Eqs.~(1) to~(5)".
%% + To refer to multiple things at once, combine them all into one:
%%   `\cref{eq2,eq1,eq3,eq5,thm2,def1}` produces:
%%   "Eqs.~(1) to~(3) and~(5), Theorem~5, and Definition~1".

%%
%% Math fonts, unicode symbols, and new theorem environments
%%
\usepackage{unicode-math}
\setsansfont{XITSMath-Regular.otf}
\setmathfont{XITSMath-Regular.otf}

\usepackage{macros}
