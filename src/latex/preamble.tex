\usepackage{longtable}
\usepackage[margin=2.5cm]{geometry}
\usepackage{float}
\floatstyle{boxed}
\restylefloat{figure}

\usepackage{iohk}
\usepackage{agda-latex-macros}
\usepackage{hyperref}
\hypersetup{
    colorlinks=true,
    linkcolor=blue,
    urlcolor=blue
 }
\usepackage[links]{agda}
\setlength{\mathindent}{10pt}


\usepackage[utf8]{inputenc} % For UTF-8 encoding (recommended)
\usepackage[style=authoryear, backend=biber]{biblatex} % use biblatex
\addbibresource{references.bib}

\usepackage{cleveref}
\usepackage{xparse}

\usepackage{fontspec}
\newcommand\agdaFont{StrippedJuliaMono}
\newcommand\agdaFontOptions{
Path=fonts/,
Extension=.ttf,
UprightFont=*-Regular,
BoldFont=*-Bold,
ItalicFont=*-RegularItalic,
BoldItalicFont=*-BoldItalic,
%% Scale=MatchLowercase
Scale=0.80
%% Scale=MatchUppercase
}
\newfontfamily{\AgdaSerifFont}{\agdaFont}[\agdaFontOptions]
\newfontfamily{\AgdaSansSerifFont}{\agdaFont}[\agdaFontOptions]
\newfontfamily{\AgdaTypewriterFont}{\agdaFont}[\agdaFontOptions]
\renewcommand{\AgdaFontStyle}[1]{{\AgdaSansSerifFont{}#1}}
\renewcommand{\AgdaKeywordFontStyle}[1]{{\AgdaSansSerifFont{}#1}}
\renewcommand{\AgdaStringFontStyle}[1]{{\AgdaTypewriterFont{}#1}}
\renewcommand{\AgdaCommentFontStyle}[1]{{\AgdaTypewriterFont{}#1}}
\renewcommand{\AgdaBoundFontStyle}[1]{{\emph{\AgdaTypewriterFont{}#1}}}

% Define the \hldiff macro to highlight text with a yellow background
\newcommand{\hldiff}[1]{\colorbox{yellow}{#1}}

% Math fonts
\usepackage{unicode-math}
\setsansfont{XITSMath-Regular.otf}
\setmathfont{XITSMath-Regular.otf}

\newcommand{\N}{\ensuremath{\mathbb{N}}}

%%
%% Functions
%%
\newcommand{\outs}[1]{\fun{outs}~ \var{#1}}
\newcommand{\balance}[1]{\fun{balance}~ \var{#1}}
\newcommand{\txfee}[1]{\fun{txfee}~ \var{#1}}

\newtheorem{property}{Property}[section]

%% -- DEFINITIONS -----------------------------------------------------------
%% Set the desired typeface of defined terms here and, at the
%% first occurrence of such a term, enclose it in `\defn{...}`.
\newcommand{\defn}[1]{\textit{#1}}   %  defined terms are typeset in italics
%% \newcommand{\defn}[1]{\textbf{#1}}   %  defined terms are typeset in bold
\newunicodechar{ˢ}{\ensuremath{}}

\newcommand{\repourl}{https://github.com/IntersectMBO/formal-ledger-specifications}
\newcommand{\modulenote}[1]{Code in this section is part of the #1
module of the \href{\repourl}{formal ledger specification}.}
\newcommand{\modulefootnotetext}[1]{\footnotetext{\modulenote{#1}}}

\ExplSyntaxOn
\NewDocumentCommand\hrefCIP{m}{%
  \clist_if_in:nnTF {0069,1694} {#1}
  {\href{\detokenize{https://cips.cardano.org/cip/CIP-#1}}{CIP-#1}}%
  {\msg_warning:nnn {agda-latex-macros} {invalid-cip} {#1}
    CIP-#1 % Still prints the text but without a hyperlink
  }%
}
\msg_new:nnn {agda-latex-macros} {invalid-cip}
  {CIP-#1 is not a known CIP. The hyperlink will not be created.}
\ExplSyntaxOff

\NewDocumentCommand{\hrefAgdaDocs}{O{} O{Agda documentation}}{%
  \href{https://agda.readthedocs.io/en/\AgdaVersion/#1}{#2}%
}
% For a link to the main Agda documentation website, just use \hrefAgdaDocs{}.
% For a link to a specific section of the Agda docs, use
% \hrefAgdaDocs[URLsuffix]{Link text}.
% For example, \hrefAgdaDocs[language/record-types.html][agda.readthedocs.io/record-types],
% produces a link to url https://agda.readthedocs.io/en/\AgdaVersion/language/record-types.html
% with link text "agda.readthedocs.io/record-types".
