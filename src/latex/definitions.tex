\section{Miscellaneous Definitions}
\label{sec:misc-defs}
To keep this document fairly self-contained, this section collects definitions of
many of the technical terms we use in presenting the Cardano ledger specification.
Some online references for this section are
\cite{www-essential-cardano-glossary,www-developers-cardano-reintro,www-ledger-academy-glossary}.


\subsection{Cardano Time Handling}
\label{sec:cardano-time-handling}
Some online references for this subsection are
\cite{www-docs-cardano-network,www-docs-cardano-time,www-iohk-blog-ouroboros,www-ledger-academy-stake-ada}.

In Cardano, the Ouroboros proof-of-stake (PoS) consensus protocol models the passage
of physical time as an infinite sequence of time slots and epochs.

\begin{definition}[slot]
  A \emph{slot} is a discrete time interval in which a single block may be produced; it
  is the fundamental time unit within the blockchain's consensus protocol.
\end{definition}
Slots should be long enough for a new block to have a good chance to reach
the next slot leader in time.  For example, the slot length in the Byron era was 20
seconds, while in Shelley and later eras it is 1 second.

Not every slot results in a new block.  Indeed, in any given slot, one or more
block-producing nodes are nominated (probabilistically based on stake distribution)
to be \textit{slot leaders} and given the opportunity to produce a new block.
For example, in Shelley and later eras, on average only 0.05 of slots will produce a
block (resulting in 20-second intervals between blocks).

\begin{definition}[epoch]
  An \emph{epoch} is a period of time, containing some number of slots, used to select
  block-producing nodes.
\end{definition}
For example, in Shelley and later eras, an epoch consists of roughly 432,000 slots (or five
days if we assume a slot length of 1 second).

\begin{definition}[slot number]
  \emph{Slot number} may refer to a slot's position within the current epoch or it
  may mean the absolute slot count since the genesis block.  The context should make
  clear which meaning is intended.
\end{definition}

\begin{definition}[block time]
  The actual time interval between blocks, or \emph{block time}, is the slot length
  (in seconds) divided by the block coefficient f, which is the expected block
  frequency (blocks per second).
\end{definition}
For example, if f is 0.05, then on average 5\% of slots contain blocks.
If the slot length is 1 second, then the block time is 20 seconds.

The parameter values mentioned in the examples above,
\begin{itemize}[noitemsep]
  \item block time = 20 seconds,
  \item slot length = 1 second,
  \item block coefficient = 0.05,
  \item slots/epoch = 432,000,
\end{itemize}
% \begin{align*}
%   \text{block time}        &= 20~\text{seconds},\\
%   \text{slot length}       &= 1~\text{second},\\
%   \text{block coefficient} &= 0.05,\\
%   \text{slots/epoch}       &= 432,000,
% \end{align*}
are unlikely to change in the short-term.  However, the longer term plan is to replace the current Ouroboros  protocol with Ouroboros Chronos, which addresses timekeeping challenges by providing the first
high-resilience cryptographic time source based on blockchain technology (see~\textcite{www-iohk-blog-ouroboros}).
