\section{Notation}
\label{sec:notation}
This section introduces some of the notation we use in this document and in our Agda formalization.

\begin{description}
\item[Propositions, sets and types.] In this document the abstract notions of ``set''
and ``type'' are essentially the same, despite having different formal definitions
in our Agda code. We represent sets as a special type, which we denote by
\AgdaDatatype{Set}~\AgdaBound{A}, for \AgdaBound{A} an arbitrary type.
(See \cref{sec:sets-maps} for details and~\textcite[Ch.~19]{NPS:1990-open} for
background.)
Agda denotes the primitive notion of type by \AgdaPrimitive{Set}.  To avoid confusion,
throughout this document and in our Agda code we call this primitive \Type{},
reserving the name \AgdaDatatype{Set} for our set type.
All of our sets are finite, and when we need to convert
a list \AgdaBound{l} to its set of elements, we write \fromList{}~\AgdaBound{l}.
\item[Lists] We use the notation \AgdaBound{a}~\AgdaInductiveConstructor{∷}~\AgdaBound{as} for
  the list with \textit{head} \AgdaBound{a} and \textit{tail} \AgdaBound{as};
  \AgdaInductiveConstructor{[]} denotes the empty list, and
  \AgdaBound{l}~\AgdaFunction{∷ʳ}~\AgdaBound{x} appends the element \AgdaBound{x}
  to the end of the list \AgdaBound{l}.
\item[Sums and products.] The sum (or disjoint union, coproduct, etc.) of \AgdaBound{A} and
  \AgdaBound{B} is denoted by \AgdaBound{A} \coproduct{} \AgdaBound{B}, and their product
  is denoted by \AgdaBound{A} \agdatimes{} \AgdaBound{B}. The projection functions from products
  are denoted \fst{} and \snd{}, and the injections are denoted \inl{} and \inr{} respectively. The
  properties whether an element of a coproduct is in the left or right component are called
  \isInl{} and \isInr{}.
\item[Addition of map values.] The expression
\AgdaFunction{∑[}\AgdaSpace{}%
\AgdaBound{x}\AgdaSpace{}%
\AgdaFunction{←}\AgdaSpace{}%
\AgdaFunction{m}\AgdaSpace{}%
\AgdaFunction{]}\AgdaSpace{}%
\AgdaFunction{f}\AgdaSpace{}%
\AgdaBound{x}
denotes the sum of the values obtained by applying the function \AgdaFunction{f} to the values
of the map \AgdaFunction{m}.
\item[Record types] are explained in \cref{sec:appendix-agda-essentials}.
\item[Postfix projections.] Projections can be written using postfix notation. For example, we may
  write \AgdaBound{x}\AgdaSpace{}\AgdaSymbol{.}\AgdaField{proj₁} instead of
  \AgdaField{proj₁}\AgdaSpace{}\AgdaBound{x}.
\item[Restriction, corestriction and complements.] The restriction of a function or map
  \AgdaBound{f} to some domain \AgdaBound{A} is denoted by \AgdaBound{f}~\AgdaFunction{|}~\AgdaBound{A},
  and the restriction to the complement of \AgdaBound{A} is written
  \AgdaBound{f}~\AgdaFunction{|}~\AgdaBound{A}~\AgdaFunction{ᶜ}. Corestriction or range restriction is
  denoted similarly, except that \AgdaFunction{|} is replaced by \corestriction{}.
\item[Inverse image.] The expression \AgdaBound{m}~\AgdaFunction{⁻¹}~\AgdaBound{B} denotes the
  inverse image of the set \AgdaBound{B} under the map \AgdaBound{m}.
\item[Left-biased union.] For maps \AgdaBound{m} and \AgdaBound{m'}, we write
  \AgdaBound{m}~\AgdaFunction{∪ˡ}~\AgdaBound{m'} for their left-biased union. This means that
  key-value pairs in \AgdaBound{m} are guaranteed to be in the union, while key-value pairs in
  \AgdaBound{m'} will be in the union if and only if the keys don't collide.
\item[Map addition.] For maps \AgdaBound{m} and \AgdaBound{m'}, we write
  \AgdaBound{m}~\AgdaFunction{∪⁺}~\AgdaBound{m'} for their union, where keys that appear
  in both maps have their corresponding values added.
\item[Mapping a partial function.] A \textit{partial function} is a function on \AgdaBound{A} which
  may not be defined for all elements of \AgdaBound{A}. We denote such a function by
  \AgdaBound{f}~:~\AgdaBound{A}~⇀~\AgdaBound{B}.  If we happen to know that the function is
  \textit{total} (defined for all elements of \AgdaBound{A}), then we write
  \AgdaBound{f}~:~\AgdaBound{A}~→~\AgdaBound{B}.
  The \mapPartial operation takes such a function \AgdaBound{f}
  and a set \AgdaBound{S} of elements of \AgdaBound{A} and applies \AgdaBound{f} to the elements
  of \AgdaBound{S} at which it is defined; the result is the set
  \(\{\text{\AgdaBound{f}~\AgdaBound{x}} ∣ \text{\AgdaBound{x}}~∈~\text{\AgdaBound{S} and
  \AgdaBound{f} is defined at \AgdaBound{x}}\}\).
\item[The \AgdaDatatype{Maybe} type]
represents an optional value and can either be
\AgdaInductiveConstructor{just}\AgdaSpace{}\AgdaBound{x}
(indicating the presence of a value, \AgdaBound{x}) or \AgdaInductiveConstructor{nothing}
(indicating the absence of a value).  If \AgdaBound{x} has type \AgdaDatatype{X}, then
\AgdaInductiveConstructor{just}\AgdaSpace{}\AgdaBound{x} has type
\AgdaDatatype{Maybe}\AgdaSpace{}\AgdaDatatype{X}.

The symbol~\AgdaDatatype{\sim} denotes (pseudo)equality of two values \AgdaBound{x} and
\AgdaBound{y} of type~\AgdaDatatype{Maybe}\AgdaSpace{}\AgdaDatatype{X}: if
\AgdaBound{x} is of the form
\AgdaInductiveConstructor{just}\AgdaSpace{}\AgdaBound{x'} and \AgdaBound{y} is
of the form \AgdaInductiveConstructor{just}\AgdaSpace{}\AgdaBound{y'}, then
\AgdaBound{x'} and \AgdaBound{y'} have to be equal. Otherwise, they are
considered ``equal''.
\item[The unit type] \AgdaRecord{⊤} has a single inhabitant \AgdaInductiveConstructor{tt} and may be thought
of as a type that carries no information; it is useful for signifying the completion of an action, the
presence of a trivial value, a trivially satisfied requirement, etc.
\end{description}

\subsection{Superscripts and Other Special Notations}
\label{sec:superscripts-other-special-notation}
In the current version of this specification, superscript letters are
heavily used for things such as disambiguations or type
conversions. These are essentially meaningless, only present for
technical reasons and can safely be ignored. However there are the
two exceptions:
\begin{itemize}
\item \AgdaFunction{∪ˡ} for left-biased union
\item \AgdaFunction{ᶜ} in the context of set restrictions, where it indicates the complement
\end{itemize}
Also, non-letter superscripts do carry meaning.\footnote{At some point in the future we
  hope to be able to remove all those non-essential superscripts.  Since we prefer doing
  this by changing the Agda source code instead of via hiding them in this document, this
  is a non-trivial problem that will take some time to address.}

Finally, there are some \AgdaFunction{?} and \AgdaFunction{¿} operations.
These relate to decision procedures and can also safely be ignored.\footnote{We
  plan on refactoring the code so that these special symbols will also disappear
  from this document.}
