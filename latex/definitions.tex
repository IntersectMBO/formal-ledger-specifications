\section{Miscellaneous Definitions}
\label{sec:misc-defs}
To keep this document fairly self-contained, this section collects definitions of
many of the technical terms we use in presenting the Cardano ledger specification.
This collection is far from comprehensive and the reader may wish to consult some of
the following online resources to help fill in any gaps:

\begin{itemize}
  \item
        \href{https://www.ledger.com/academy/glossary}%
             {Glossary from ledger.com} \textcite{www-ledger-academy-glossary}

  \item 
        \href{https://www.essentialcardano.io/glossary?sort=alphabetical}%
             {Glossary from essentialcardano.io} \textcite{www-essential-cardano-glossary}
  \item
        \href{https://docs.cardano.org/about-cardano/explore-more/time}%
             {Time handling on Cardano} \textcite{www-docs-cardano-time};
  \item
        \href{https://developers.cardano.org/docs/operate-a-stake-pool/introduction-to-cardano/}%
             {Cardano network: operate a stake pool} \textcite{www-docs-cardano-network};
  \item
        \href{https://developers.cardano.org/docs/operate-a-stake-pool/introduction-to-cardano/}%
             {(Re)introduction to Cardano} \textcite{www-developers-cardano-reintro};
  \item
        \href{https://iohk.io/en/blog/posts/2021/10/27/ouroboros-chronos-provides-the-first-high-resilience-cryptographic-time-source-based-on-blockchain/}%
             {Ouroboros Chronos provides the first high-resilience, cryptographic time source based on blockchain technology} \textcite{www-iohk-blog-ouroboros};
  \item
        \href{https://www.ledger.com/academy/cardano-staking-how-to-stake-ada}%
             {Cardano Staking: How To Stake ADA} \textcite{www-ledger-academy-stake-ada}.
\end{itemize}

\subsection{Cardano Time Handling}
\label{sec:cardano-time-handling}

In Cardano, the Ouroboros proof-of-stake (PoS) consensus protocol models the passage
of physical time as an infinite sequence of time slots and epochs.

\begin{definition}[slot]
  A \emph{slot} is a discrete time interval in which a single block may be produced; it
  is the fundamental time unit within the blockchain's consensus protocol.
\end{definition}
Slots should be long enough for a new block to have a good chance to reach
the next slot leader in time.  For example, the slot length in the Byron era was 20
seconds, while in Shelley and later eras it is 1 second.

Not every slot results in a new block.  Indeed, in any given slot, one or more
block-producing nodes are nominated (probabilistically based on stake distribution)
to be \textit{slot leaders} and given the opportunity to produce a new block.
For example, in Shelley and later eras, on average only 0.05 of slots will produce a
block (resulting in 20-second intervals between blocks).

\begin{definition}[epoch]
  An \emph{epoch} is a period of time, containing some number of slots, used to select
  block-producing nodes.
\end{definition}
For example, in Shelley and later eras, an epoch consists of roughly 432,000 slots (or five
days if we assume a slot length of 1 second).

\begin{definition}[slot number]
  \emph{Slot number} may refer to a slot's position within the current epoch or it
  may mean the absolute slot count since the genesis block.  The context should make
  clear which meaning is intended.
\end{definition}

\begin{definition}[block time]
  The actual time interval between blocks, or \emph{block time}, is the slot length
  (in seconds) divided by the block coefficient f, which is the expected block
  frequency (blocks per second).
\end{definition}
For example, if f is 0.05, then on average 5\% of slots contain blocks.
If the slot length is 1 second, then the block time is 20 seconds.

The parameter values mentioned in the examples above,
\begin{itemize}[noitemsep]
  \item block time = 20 seconds,
  \item slot length = 1 second,
  \item block coefficient = 0.05,
  \item slots/epoch = 432,000,
\end{itemize}
are unlikely to change in the short-term.  However, the longer term plan is to replace the current Ouroboros  protocol with Ouroboros Chronos, which addresses timekeeping challenges by providing the first
high-resilience cryptographic time source based on blockchain technology (see~\textcite{www-iohk-blog-ouroboros}).
