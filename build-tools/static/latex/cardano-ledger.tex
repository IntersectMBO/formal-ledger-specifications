\documentclass[11pt,a4paper,dvipsnames]{article}

\usepackage{environ}
\NewEnviron{NoConway}{%
\BODY
}
\NewEnviron{Conway}{%
}

\usepackage{longtable}
\usepackage[margin=2.5cm]{geometry}
\usepackage{float}
\floatstyle{boxed}
\restylefloat{figure}

\usepackage{iohk}
\usepackage{agda-latex-macros}
\usepackage{hyperref}
\hypersetup{
    colorlinks=true,
    linkcolor=blue,
    urlcolor=blue
 }
\usepackage[links]{agda}
\setlength{\mathindent}{10pt}

\usepackage{fontspec}
\newcommand\agdaFont{StrippedJuliaMono}
\newcommand\agdaFontOptions{
Path=fonts/,
Extension=.ttf,
UprightFont=*-Regular,
BoldFont=*-Bold,
ItalicFont=*-RegularItalic,
BoldItalicFont=*-BoldItalic,
%% Scale=MatchLowercase
Scale=0.80
%% Scale=MatchUppercase
}
\newfontfamily{\AgdaSerifFont}{\agdaFont}[\agdaFontOptions]
\newfontfamily{\AgdaSansSerifFont}{\agdaFont}[\agdaFontOptions]
\newfontfamily{\AgdaTypewriterFont}{\agdaFont}[\agdaFontOptions]
\renewcommand{\AgdaFontStyle}[1]{{\AgdaSansSerifFont{}#1}}
\renewcommand{\AgdaKeywordFontStyle}[1]{{\AgdaSansSerifFont{}#1}}
\renewcommand{\AgdaStringFontStyle}[1]{{\AgdaTypewriterFont{}#1}}
\renewcommand{\AgdaCommentFontStyle}[1]{{\AgdaTypewriterFont{}#1}}
\renewcommand{\AgdaBoundFontStyle}[1]{{\emph{\AgdaTypewriterFont{}#1}}}

% Define the \hldiff macro to highlight text with a yellow background
\newcommand{\hldiff}[1]{\colorbox{yellow}{#1}}

% Math fonts
\usepackage{unicode-math}
\setsansfont{XITSMath-Regular.otf}
\setmathfont{XITSMath-Regular.otf}

\newcommand{\N}{\ensuremath{\mathbb{N}}}

%%
%% Functions
%%
\newcommand{\txins}[1]{\fun{txins}~ \var{#1}}
\newcommand{\txouts}[1]{\fun{txouts}~ \var{#1}}
\newcommand{\txid}[1]{\fun{txid}~ \var{#1}}
\newcommand{\outs}[1]{\fun{outs}~ \var{#1}}
\newcommand{\balance}[1]{\fun{balance}~ \var{#1}}
\newcommand{\txfee}[1]{\fun{txfee}~ \var{#1}}

\newtheorem{property}{Property}[section]

%% -- DEFINITIONS -----------------------------------------------------------
%% Set the desired typeface of defined terms here and, at the
%% first occurrence of such a term, enclose it in `\defn{...}`.
\newcommand{\defn}[1]{\textit{#1}}   %  defined terms are typeset in italics
%% \newcommand{\defn}[1]{\textbf{#1}}   %  defined terms are typeset in bold
\newunicodechar{ˢ}{\ensuremath{}}


\begin{document}

\tableofcontents

\includeAgda{Ledger/Introduction}
\section{Notation}
\label{sec:notation}
This section introduces some of the notation we use in this document and in our Agda formalization.

\begin{description}
\item[Propositions, sets and types.] In this document the abstract notions of ``set''
and ``type'' are essentially the same, despite having different formal definitions
in our Agda code. We represent sets as a special type, which we denote by
\AgdaDatatype{Set}~\AgdaBound{A}, for \AgdaBound{A} an arbitrary type.
(See Section~\ref{sec:sets-maps} for details and~\cite[Chapter 19]{NPS:1990-open} for
background.)
Agda denotes the primitive notion of type by \AgdaPrimitive{Set}.  To avoid confusion,
throughout this document and in our Agda code we call this primitive \Type,
reserving the name \AgdaDatatype{Set} for our set type.
All of our sets are finite, and when we need to convert
a list \AgdaBound{l} to its set of elements, we write \fromList~\AgdaBound{l}.
\item[Lists] We use the notation \AgdaBound{a}~\AgdaInductiveConstructor{∷}~\AgdaBound{as} for
  the list with \textit{head} \AgdaBound{a} and \textit{tail} \AgdaBound{as};
  \AgdaInductiveConstructor{[]} denotes the empty list, and
  \AgdaBound{l}~\AgdaFunction{∷ʳ}~\AgdaBound{x} appends the element \AgdaBound{x}
  to the end of the list \AgdaBound{l}.
\item[Sums and products.] The sum (or disjoint union, coproduct, etc.) of \AgdaBound{A} and
  \AgdaBound{B} is denoted by \AgdaBound{A} \coproduct \AgdaBound{B}, and their product
  is denoted by \AgdaBound{A} \agdatimes \AgdaBound{B}. The projection functions from products
  are denoted \fst and \snd, and the injections are denoted \inl and \inr respectively. The
  properties whether an element of a coproduct is in the left or right component are called
  \isInl and \isInr.
\item[Addition of map values.] The expression
\AgdaFunction{∑[}\AgdaSpace{}%
\AgdaBound{x}\AgdaSpace{}%
\AgdaFunction{←}\AgdaSpace{}%
\AgdaFunction{m}\AgdaSpace{}%
\AgdaFunction{]}\AgdaSpace{}%
\AgdaFunction{f}\AgdaSpace{}%
\AgdaBound{x}
denotes the sum of the values obtained by applying the function \AgdaFunction{f} to the values
of the map \AgdaFunction{m}.
\item[Record types] are explained in Appendix~\ref{sec:appendix-agda-essentials}.
\item[Postfix projections.] Projections can be written using postfix notation. For example, we may
  write \AgdaBound{x}\AgdaSpace{}\AgdaSymbol{.}\AgdaField{proj₁} instead of
  \AgdaField{proj₁}\AgdaSpace{}\AgdaBound{x}.
\item[Restriction, corestriction and complements.] The restriction of a function or map
  \AgdaBound{f} to some domain \AgdaBound{A} is denoted by \AgdaBound{f}~\AgdaFunction{|}~\AgdaBound{A},
  and the restriction to the complement of \AgdaBound{A} is written
  \AgdaBound{f}~\AgdaFunction{|}~\AgdaBound{A}~\AgdaFunction{ᶜ}. Corestriction or range restriction is
  denoted similarly, except that \AgdaFunction{|} is replaced by \corestriction.
\item[Inverse image.] The expression \AgdaBound{m}~\AgdaFunction{⁻¹}~\AgdaBound{B} denotes the
  inverse image of the set \AgdaBound{B} under the map \AgdaBound{m}.
\item[Left-biased union.] For maps \AgdaBound{m} and \AgdaBound{m'}, we write
  \AgdaBound{m}~\AgdaFunction{∪ˡ}~\AgdaBound{m'} for their left-biased union. This means that
  key-value pairs in \AgdaBound{m} are guaranteed to be in the union, while key-value pairs in
  \AgdaBound{m'} will be in the union if and only if the keys don't collide.
\item[Map addition.] For maps \AgdaBound{m} and \AgdaBound{m'}, we write
  \AgdaBound{m}~\AgdaFunction{∪⁺}~\AgdaBound{m'} for their union, where keys that appear
  in both maps have their corresponding values added.
\item[Mapping a partial function.] A \textit{partial function} is a function on \AgdaBound{A} which
  may not be defined for all elements of \AgdaBound{A}. We denote such a function by
  \AgdaBound{f}~:~\AgdaBound{A}~⇀~\AgdaBound{B}.  If we happen to know that the function is
  \textit{total} (defined for all elements of \AgdaBound{A}), then we write
  \AgdaBound{f}~:~\AgdaBound{A}~→~\AgdaBound{B}.
  The \mapPartial operation takes such a function \AgdaBound{f}
  and a set \AgdaBound{S} of elements of \AgdaBound{A} and applies \AgdaBound{f} to the elements
  of \AgdaBound{S} at which it is defined; the result is the set
  \(\{\text{\AgdaBound{f}~\AgdaBound{x}} ∣ \text{\AgdaBound{x}}~∈~\text{\AgdaBound{S} and
  \AgdaBound{f} is defined at \AgdaBound{x}}\}\).
\item[The \AgdaDatatype{Maybe} type]
represents an optional value and can either be
\AgdaInductiveConstructor{just}\AgdaSpace{}\AgdaBound{x}
(indicating the presence of a value, \AgdaBound{x}) or \AgdaInductiveConstructor{nothing}
(indicating the absence of a value).  If \AgdaBound{x} has type \AgdaDatatype{X}, then
\AgdaInductiveConstructor{just}\AgdaSpace{}\AgdaBound{x} has type
\AgdaDatatype{Maybe}\AgdaSpace{}\AgdaDatatype{X}.

The symbol~\AgdaDatatype{≡?} denotes (pseudo)equality of two values \AgdaBound{x} and
\AgdaBound{y} of type~\AgdaDatatype{Maybe}\AgdaSpace{}\AgdaDatatype{X}: if
\AgdaBound{x} is of the form
\AgdaInductiveConstructor{just}\AgdaSpace{}\AgdaBound{x'} and \AgdaBound{y} is
of the form \AgdaInductiveConstructor{just}\AgdaSpace{}\AgdaBound{y'}, then
\AgdaBound{x'} and \AgdaBound{y'} have to be equal. Otherwise, they are
considered ``equal''.
\item[The unit type] \AgdaRecord{⊤} has a single inhabitant \AgdaInductiveConstructor{tt} and may be thought
of as a type that carries no information; it is useful for signifying the completion of an action, the
presence of a trivial value, a trivially satisfied requirement, etc.
\end{description}

\includeAgda{Ledger/Conway/Crypto}
\includeAgda{Ledger/Conway/BaseTypes}
\includeAgda{Ledger/Conway/TokenAlgebra}
\includeAgda{Ledger/Conway/Address}
\includeAgda{Ledger/Conway/Script}
\includeAgda{Ledger/Conway/PParams}
\includeAgda{Ledger/Conway/Fees}
\includeAgda{Ledger/Conway/GovernanceActions}
\includeAgda{Ledger/Conway/Transaction}
\includeAgda{Ledger/Conway/Utxo}
\includeAgda{Ledger/Conway/Utxow}
\includeAgda{Ledger/Conway/Gov}
\includeAgda{Ledger/Conway/Certs}
\includeAgda{Ledger/Conway/Ledger}
\includeAgda{Ledger/Conway/Enact}
\includeAgda{Ledger/Conway/Ratify}
\includeAgda{Ledger/Conway/Rewards}
\includeAgda{Ledger/Conway/Epoch}
\includeAgda{Ledger/Conway/Chain}

\section{Properties}
This section presents the properties of the ledger that we have formally proved in
Agda or plan to do so in the near future.  We indicate in which Agda module each
property is formally stated and (possibly) proved. A ``Claim'' is a property that is
not yet proved, while a ``Theorem'' is one for which we have a formal proof.

\subsection{Preservation of Value}
\inputAgda{Ledger/Conway/Ledger/Properties/PoV}
\inputAgda{Ledger/Conway/Utxo/Properties/PoV}
\inputAgda{Ledger/Conway/Certs/Properties/PoV}
\inputAgda{Ledger/Conway/Certs/Properties/PoVLemmas}

\subsection{Invariance Properties}
To say that a predicate \ab{P} is an \textit{invariant} of a transition rule
means the following: if the transition rule relates states \ab{s} and \ab{s'} and if
\ab{P} holds at state~\ab{s}, then \ab{P} holds at state \ab{s'}.

\inputAgda{Ledger/Conway/Chain/Properties/CredDepsEqualDomRwds}
\inputAgda{Ledger/Conway/Chain/Properties/PParamsWellFormed}

\subsubsection{Governance Action Deposits Match}
\Cref{thm:ChainGovDepsMatch,lem:LedgerGovDepsMatch,lem:EpochGovDepsMatch} assert that
a certain predicate is an invariant of the \CHAIN{}, \LEDGER{}, and \EPOCH{} rules, respectively.
Given a ledger state \ab{s}, we focus on deposits in the
\UTxOState{} of \ab{s} that are \GovActionDeposit{}s and we compare that set of
deposits with the \GovActionDeposit{}s of the \GovState{} of \ab{s}.
When these two sets are the same, we write \AgdaFunction{govDepsMatch}~\ab{s} and say
the \AgdaFunction{govDepsMatch} relation holds for \ab{s}.
Formally, the \AgdaFunction{govDepsMatch} predicate is defined as follows:
\inputAgda{Ledger/Conway/Ledger/Properties}

The assertion,
``the \AgdaFunction{govDepsMatch} relation is an invariant of the \LEDGER{} rule,''
means the following:  if \govDepsMatch{}~\ab{s} and
\ab{s}~\AgdaDatatype{⇀⦇}~\ab{tx}~\AgdaDatatype{,LEDGER⦈}~\ab{s'}, then
\govDepsMatch{}~\ab{s'}.

\inputAgda{Ledger/Conway/Chain/Properties/GovDepsMatch}
\inputAgda{Ledger/Conway/Ledger/Properties/GovDepsMatch}
\inputAgda{Ledger/Conway/Epoch/Properties/GovDepsMatch}

\subsection{Minimum Spending Conditions}
\inputAgda{Ledger/Conway/Utxo/Properties/MinSpend}

\subsection{Miscellaneous Properties}
\inputAgda{Ledger/Conway/GovernanceActions/Properties/ChangePPGroup}
\inputAgda{Ledger/Conway/Chain/Properties/EpochStep}
\inputAgda{Ledger/Conway/Epoch/Properties/ConstRwds}
\inputAgda{Ledger/Conway/Epoch/Properties/NoPropSameDReps}
\inputAgda{Ledger/Conway/Certs/Properties/VoteDelegsVDeleg}

\clearpage

\addcontentsline{toc}{section}{References}

\printbibliography

\clearpage

\appendix

\section{Definitions}
\label{sec:definitions}
To keep this document somewhat self-contained, we define some technical terms that
arise when defining and describing the Cardano ledger.
This is not meant to be comprehensive and the reader may wish to consult
online resources to fill in any gaps.  Here are a few such resources that might be
helpful.

\begin{itemize}
  \item
        \href{https://docs.cardano.org/}%
             {Cardano Docs}~\cite{www-docs-cardano};
  \item
        \href{https://developers.cardano.org/docs/operate-a-stake-pool/introduction-to-cardano/}%
             {(Re)introduction to Cardano}~\cite{www-developers-cardano-reintro};
  \item
        \href{https://iohk.io/en/blog/posts/2021/10/27/ouroboros-chronos-provides-the-first-high-resilience-cryptographic-time-source-based-on-blockchain/}%
             {Ouroboros Chronos blog post}~\cite{www-iohk-blog-ouroboros};
  \item
        \href{https://www.ledger.com/academy/cardano-staking-how-to-stake-ada}%
             {Cardano Staking: How To Stake ADA}~\cite{www-ledger-academy-stake-ada};
  \item
        \href{https://cardano.org/docs/glossary#cardano-glossary}%
             {Glossary from cardano.org}~\cite{www-cardano-glossary};
  \item
        \href{https://www.ledger.com/academy/glossary}%
             {Glossary from ledger.com}~\cite{www-ledger-academy-glossary};
  \item
        \href{https://www.essentialcardano.io/glossary?sort=alphabetical}%
             {Glossary from essentialcardano.io}~\cite{www-essential-cardano-glossary}.
\end{itemize}

\subsection{Cardano Time Handling}
\label{sec:cardano-time-handling}

For more details, see the
\href{https://docs.cardano.org/about-cardano/explore-more/time}%
     {Time handling on Cardano} section of~\cite{www-docs-cardano-time}.

In Cardano, the Ouroboros proof-of-stake (PoS) consensus protocol models the passage
of physical time as an infinite sequence of time slots and epochs.

\begin{itemize}
\item[\textbf{block time}]
  The actual time interval between blocks, or \emph{block time}, is the slot length
  (in seconds) divided by the block coefficient f, which is the expected block
  frequency (blocks per second).
  For example, if f is 0.05, then on average 5\% of slots contain blocks.
  If the slot length is 1 second, then the block time is 20 seconds.

\item[\textbf{epoch}]
  An \emph{epoch} is a period of time, containing some number of slots, used to select
  block-producing nodes.
  For example, in Shelley and later eras, an epoch consists of roughly 432,000 slots (or five
  days if we assume a slot length of 1 second).

\item[\textbf{genesis block}]
  The \emph{genesis block} of Cardano was created on the 23rd of September 2017. As the
  first block in the blockchain, it set the foundation for the network, it does not
  reference any previous blocks, and it generated the initial supply of Ada.

\item[\textbf{slot}]
  A \emph{slot} is a discrete time interval in which a single block may be produced; it
  is the fundamental time unit within the blockchain's consensus protocol.
  Slots should be long enough for a new block to have a good chance to reach the next
  slot leader in time.  For example, the slot length in the Byron era was 20
  seconds, while in Shelley and later eras it is 1 second.
  Not every slot results in a new block.  Indeed, in any given slot, one or more
  block-producing nodes are nominated (probabilistically based on stake distribution)
  to be \textit{slot leaders} and given the opportunity to produce a new block.
  For example, in Shelley and later eras, on average only 0.05 of slots will produce a
  block (resulting in 20-second intervals between blocks).
  \emph{Slot number} may refer to a slot's position within the current epoch or it
  may mean the absolute slot count since the genesis block.  The context should make
  clear which meaning is intended.
\end{itemize}

The parameter values mentioned in the examples above,
\begin{itemize}[noitemsep]
  \item block time = 20 seconds,
  \item slot length = 1 second,
  \item block coefficient = 0.05,
  \item slots/epoch = 432,000,
\end{itemize}
are unlikely to change in the short-term.  However, the longer term plan is to
replace the current Ouroboros  protocol with Ouroboros Chronos, which addresses
timekeeping challenges by providing the first high-resilience cryptographic time
source based on blockchain technology (see~\textcite{www-iohk-blog-ouroboros}).

\inputAgda{EssentialAgda}
\section{Bootstrapping EnactState}
\label{sec:conway-bootstrap-enact}

To form an \EnactState{}, some governance action IDs need to be
provided. However, at the time of the initial hard fork into Conway
there are no such previous actions. There are effectively two ways to
solve this issue:

\begin{itemize}
\item populate those fields with IDs chosen in some manner (e.g. random, all zeros, etc.), or
\item add a special value to the types to indicate this situation.
\end{itemize}

In the Haskell implementation the latter solution was chosen. This
means that everything that deals with \GovActionID{} needs to be aware
of this special case and handle it properly.

This specification could have mirrored this choice, but it is not
necessary here: since it is already necessary to assume the absence of
hash-collisions (specifically first pre-image resistance) for various
properties, we could pick arbitrary initial values to mirror this
situation. Then, since \GovActionID{} contains a hash, that arbitrary
initial value behaves just like a special case.

\section{Bootstrapping the Governance System}
\label{sec:conway-bootstrap-gov}

As described in \hrefCIP{1694}, the governance system needs to be
bootstrapped. During the bootstrap period, the following changes will
be made to the ledger described in this document.

\begin{itemize}
\item Transactions containing any proposal except \TriggerHF{},
      \ChangePParams{} or \Info{} will be rejected.
\item Transactions containing a vote other than a \CC{} vote,
      a \SPO{} vote on a \TriggerHF{} action or any vote on an \Info{}
      action will be rejected.
\item \Qfour{}, \Pfive{} and \Qfivee{} are set to $0$.
\item An SPO that does not vote is assumed to have voted \abstain{}.
\end{itemize}

This allows for a governance mechanism similar to the old, Shelley-era
governance during the bootstrap phase, where the constitutional
committee is mostly in charge~\parencite{shelley-delegation-design}.
These restrictions will be removed during a subsequent hard fork,
once enough DRep stake is present in the system to properly govern
and secure itself.


\end{document}



