\documentclass[11pt,a4paper,dvipsnames]{article}

\usepackage{environ}
\NewEnviron{NoConway}{%
\BODY
}
\NewEnviron{Conway}{%
}

\usepackage{longtable}
\usepackage[margin=2.5cm]{geometry}
\usepackage{float}
\floatstyle{boxed}
\restylefloat{figure}

\usepackage{iohk}
\usepackage{agda-latex-macros}
\usepackage{hyperref}
\hypersetup{
    colorlinks=true,
    linkcolor=blue,
    urlcolor=blue
 }
\usepackage[links]{agda}
\setlength{\mathindent}{10pt}

\usepackage{fontspec}
\newcommand\agdaFont{StrippedJuliaMono}
\newcommand\agdaFontOptions{
Path=fonts/,
Extension=.ttf,
UprightFont=*-Regular,
BoldFont=*-Bold,
ItalicFont=*-RegularItalic,
BoldItalicFont=*-BoldItalic,
%% Scale=MatchLowercase
Scale=0.80
%% Scale=MatchUppercase
}
\newfontfamily{\AgdaSerifFont}{\agdaFont}[\agdaFontOptions]
\newfontfamily{\AgdaSansSerifFont}{\agdaFont}[\agdaFontOptions]
\newfontfamily{\AgdaTypewriterFont}{\agdaFont}[\agdaFontOptions]
\renewcommand{\AgdaFontStyle}[1]{{\AgdaSansSerifFont{}#1}}
\renewcommand{\AgdaKeywordFontStyle}[1]{{\AgdaSansSerifFont{}#1}}
\renewcommand{\AgdaStringFontStyle}[1]{{\AgdaTypewriterFont{}#1}}
\renewcommand{\AgdaCommentFontStyle}[1]{{\AgdaTypewriterFont{}#1}}
\renewcommand{\AgdaBoundFontStyle}[1]{{\emph{\AgdaTypewriterFont{}#1}}}

% Define the \hldiff macro to highlight text with a yellow background
\newcommand{\hldiff}[1]{\colorbox{yellow}{#1}}

% Math fonts
\usepackage{unicode-math}
\setsansfont{XITSMath-Regular.otf}
\setmathfont{XITSMath-Regular.otf}

\newcommand{\N}{\ensuremath{\mathbb{N}}}

%%
%% Functions
%%
\newcommand{\outs}[1]{\fun{outs}~ \var{#1}}
\newcommand{\balance}[1]{\fun{balance}~ \var{#1}}
\newcommand{\txfee}[1]{\fun{txfee}~ \var{#1}}

\newtheorem{property}{Property}[section]

%% -- DEFINITIONS -----------------------------------------------------------
%% Set the desired typeface of defined terms here and, at the
%% first occurrence of such a term, enclose it in `\defn{...}`.
\newcommand{\defn}[1]{\textit{#1}}   %  defined terms are typeset in italics
%% \newcommand{\defn}[1]{\textbf{#1}}   %  defined terms are typeset in bold
\newunicodechar{ˢ}{\ensuremath{}}


\begin{document}

\tableofcontents

\includeAgda{Ledger/Introduction}
\include{Notation}
\includeAgda{Ledger/Conway/Crypto}
\includeAgda{Ledger/Conway/BaseTypes}
\includeAgda{Ledger/Conway/TokenAlgebra}
\includeAgda{Ledger/Conway/Address}
\includeAgda{Ledger/Conway/Script/Base}
\includeAgda{Ledger/Conway/PParams}
\includeAgda{Ledger/Conway/Fees}
\includeAgda{Ledger/Conway/GovernanceActions}
\includeAgda{Ledger/Conway/Transaction}
\includeAgda{Ledger/Conway/Utxo}
\includeAgda{Ledger/Conway/Utxow}
\includeAgda{Ledger/Conway/Gov}
\includeAgda{Ledger/Conway/Certs}
\includeAgda{Ledger/Conway/Ledger}
\includeAgda{Ledger/Conway/Enact}
\includeAgda{Ledger/Conway/Ratify}
\includeAgda{Ledger/Conway/Rewards}
\includeAgda{Ledger/Conway/Epoch}
\includeAgda{Ledger/Conway/Chain}

\section{Properties}
This section presents the properties of the ledger that we have formally proved in
Agda or plan to do so in the near future.  We indicate in which Agda module each
property is formally stated and (possibly) proved. A ``Claim'' is a property that is
not yet proved, while a ``Theorem'' is one for which we have a formal proof.

\subsection{Preservation of Value}
\inputAgda{Ledger/Conway/Ledger/Properties/PoV}
\inputAgda{Ledger/Conway/Utxo/Properties/PoV}
\inputAgda{Ledger/Conway/Certs/Properties/PoV}
\inputAgda{Ledger/Conway/Certs/Properties/PoVLemmas}

\subsection{Invariance Properties}
To say that a predicate \ab{P} is an \textit{invariant} of a transition rule
means the following: if the transition rule relates states \ab{s} and \ab{s'} and if
\ab{P} holds at state~\ab{s}, then \ab{P} holds at state \ab{s'}.

\inputAgda{Ledger/Conway/Chain/Properties/CredDepsEqualDomRwds}
\inputAgda{Ledger/Conway/Chain/Properties/PParamsWellFormed}

\subsubsection{Governance Action Deposits Match}
\Cref{thm:ChainGovDepsMatch,lem:LedgerGovDepsMatch,lem:EpochGovDepsMatch} assert that
a certain predicate is an invariant of the \CHAIN{}, \LEDGER{}, and \EPOCH{} rules, respectively.
Given a ledger state \ab{s}, we focus on deposits in the
\UTxOState{} of \ab{s} that are \GovActionDeposit{}s and we compare that set of
deposits with the \GovActionDeposit{}s of the \GovState{} of \ab{s}.
When these two sets are the same, we write \AgdaFunction{govDepsMatch}~\ab{s} and say
the \AgdaFunction{govDepsMatch} relation holds for \ab{s}.
Formally, the \AgdaFunction{govDepsMatch} predicate is defined as follows:
\inputAgda{Ledger/Conway/Ledger/Properties}

The assertion,
``the \AgdaFunction{govDepsMatch} relation is an invariant of the \LEDGER{} rule,''
means the following:  if \govDepsMatch{}~\ab{s} and
\ab{s}~\AgdaDatatype{⇀⦇}~\ab{tx}~\AgdaDatatype{,LEDGER⦈}~\ab{s'}, then
\govDepsMatch{}~\ab{s'}.

\inputAgda{Ledger/Conway/Chain/Properties/GovDepsMatch}
\inputAgda{Ledger/Conway/Ledger/Properties/GovDepsMatch}
\inputAgda{Ledger/Conway/Epoch/Properties/GovDepsMatch}

\subsection{Minimum Spending Conditions}
\inputAgda{Ledger/Conway/Utxo/Properties/MinSpend}

\subsection{Miscellaneous Properties}
\inputAgda{Ledger/Conway/GovernanceActions/Properties/ChangePPGroup}
\inputAgda{Ledger/Conway/Chain/Properties/EpochStep}
\inputAgda{Ledger/Conway/Epoch/Properties/ConstRwds}
\inputAgda{Ledger/Conway/Epoch/Properties/NoPropSameDReps}
\inputAgda{Ledger/Conway/Certs/Properties/VoteDelegsVDeleg}

\clearpage

\addcontentsline{toc}{section}{References}

\printbibliography

\clearpage

\appendix

\section{Definitions}
\label{sec:definitions}
To keep this document somewhat self-contained, we define some technical terms that
arise when defining and describing the Cardano ledger.
This is not meant to be comprehensive and the reader may wish to consult
online resources to fill in any gaps.  Here are a few such resources that might be
helpful.

\begin{itemize}
  \item
        \href{https://docs.cardano.org/}%
             {Cardano Docs}~\cite{www-docs-cardano};
  \item
        \href{https://developers.cardano.org/docs/operate-a-stake-pool/introduction-to-cardano/}%
             {(Re)introduction to Cardano}~\cite{www-developers-cardano-reintro};
  \item
        \href{https://iohk.io/en/blog/posts/2021/10/27/ouroboros-chronos-provides-the-first-high-resilience-cryptographic-time-source-based-on-blockchain/}%
             {Ouroboros Chronos blog post}~\cite{www-iohk-blog-ouroboros};
  \item
        \href{https://www.ledger.com/academy/cardano-staking-how-to-stake-ada}%
             {Cardano Staking: How To Stake ADA}~\cite{www-ledger-academy-stake-ada};
  \item
        \href{https://cardano.org/docs/glossary#cardano-glossary}%
             {Glossary from cardano.org}~\cite{www-cardano-glossary};
  \item
        \href{https://www.ledger.com/academy/glossary}%
             {Glossary from ledger.com}~\cite{www-ledger-academy-glossary};
  \item
        \href{https://www.essentialcardano.io/glossary?sort=alphabetical}%
             {Glossary from essentialcardano.io}~\cite{www-essential-cardano-glossary}.
\end{itemize}

\subsection{Cardano Time Handling}
\label{sec:cardano-time-handling}

For more details, see the
\href{https://docs.cardano.org/about-cardano/explore-more/time}%
     {Time handling on Cardano} section of~\cite{www-docs-cardano-time}.

In Cardano, the Ouroboros proof-of-stake (PoS) consensus protocol models the passage
of physical time as an infinite sequence of time slots and epochs.

\begin{itemize}
\item[\textbf{block time}]
  The actual time interval between blocks, or \emph{block time}, is the slot length
  (in seconds) divided by the block coefficient f, which is the expected block
  frequency (blocks per second).
  For example, if f is 0.05, then on average 5\% of slots contain blocks.
  If the slot length is 1 second, then the block time is 20 seconds.

\item[\textbf{epoch}]
  An \emph{epoch} is a period of time, containing some number of slots, used to select
  block-producing nodes.
  For example, in Shelley and later eras, an epoch consists of roughly 432,000 slots (or five
  days if we assume a slot length of 1 second).

\item[\textbf{genesis block}]
  The \emph{genesis block} of Cardano was created on the 23rd of September 2017. As the
  first block in the blockchain, it set the foundation for the network, it does not
  reference any previous blocks, and it generated the initial supply of Ada.

\item[\textbf{slot}]
  A \emph{slot} is a discrete time interval in which a single block may be produced; it
  is the fundamental time unit within the blockchain's consensus protocol.
  Slots should be long enough for a new block to have a good chance to reach the next
  slot leader in time.  For example, the slot length in the Byron era was 20
  seconds, while in Shelley and later eras it is 1 second.
  Not every slot results in a new block.  Indeed, in any given slot, one or more
  block-producing nodes are nominated (probabilistically based on stake distribution)
  to be \textit{slot leaders} and given the opportunity to produce a new block.
  For example, in Shelley and later eras, on average only 0.05 of slots will produce a
  block (resulting in 20-second intervals between blocks).
  \emph{Slot number} may refer to a slot's position within the current epoch or it
  may mean the absolute slot count since the genesis block.  The context should make
  clear which meaning is intended.
\end{itemize}

The parameter values mentioned in the examples above,
\begin{itemize}[noitemsep]
  \item block time = 20 seconds,
  \item slot length = 1 second,
  \item block coefficient = 0.05,
  \item slots/epoch = 432,000,
\end{itemize}
are unlikely to change in the short-term.  However, the longer term plan is to
replace the current Ouroboros  protocol with Ouroboros Chronos, which addresses
timekeeping challenges by providing the first high-resilience cryptographic time
source based on blockchain technology (see~\textcite{www-iohk-blog-ouroboros}).

\inputAgda{EssentialAgda}
\input{ConwayBootstrapEnact}
\input{ConwayBootstrap}

\end{document}



